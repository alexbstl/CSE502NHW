%
% CSE Electronic Homework Template
% Last modified 4/20/2015 by Jeremy Buhler

\documentclass[11pt]{article}
\usepackage[left=0.7in,right=0.7in,top=0.7in,bottom=0.7in]{geometry}
\usepackage{fancyhdr} % for header
\usepackage{graphicx} % for figures
\usepackage{amsmath}  % for extended math markup
\usepackage{amssymb}
\usepackage[bookmarks=false]{hyperref} % for URL embedding
\usepackage[noend]{algpseudocode} % for pseudocode

%%%%%%%%%%%%%%%%%%%%%%%%%%%%%%%%%%%%%%%%%%%%%%%%%%%%%%%%%%%%%%%%%%%%%%
% STUDENT: modify the following fields to reflect your
% name/WUSTL Key, the current homework, and the current problem number

% Example: 
%\newcommand{\StudentName}{Jeremy Buhler}
%\newcommand{\WUSTLKey}{jbuhler}

\newcommand{\StudentName}{Alex Bernstein}
\newcommand{\WUSTLKey}{a.bernstein}
\newcommand{\HomeworkNumber}{0}
\newcommand{\ProblemNumber}{1}

%%%%%%%%%%%%%%%%%%%%%%%%%%%%%%%%%%%%%%%%%%%%%%%%%%%%%%%%%%%%%%%%%%%%%%%%
% You can pretty much leave the stuff up to the next line of %%'s alone.

% create header and footer for every page
\pagestyle{fancy}
\fancyhf{}
\lhead{\textbf{\StudentName{} (\WUSTLKey)}}
\chead{\textbf{Homework \HomeworkNumber}}
\rhead{\textbf{Problem \ProblemNumber}}
\cfoot{\thepage}

% preferred pseudocode style
\algrenewcommand{\algorithmicprocedure}{}
\algrenewcommand{\algorithmicthen}{}

% ``do { ... } while (cond)''
\algdef{SE}[DOWHILE]{Do}{doWhile}{\algorithmicdo}[1]{\algorithmicwhile\ #1}%

% ``for (x in y ... z)''
\newcommand{\ForRange}[3]{\For{#1 \textbf{in} #2 \ \ldots \ #3}}

% these are common math formatting commands that aren't defined by default
\newcommand{\union}{\cup}
\newcommand{\isect}{\cap}
\newcommand{\ceil}[1]{\ensuremath \left\lceil #1 \right\rceil}
\newcommand{\floor}[1]{\ensuremath \left\lfloor #1 \right\rfloor}

%%%%%%%%%%%%%%%%%%%%%%%%%%%%%%%%%%%%%%%%%%%%%%%%%%%%%%%%%%%%%%%%%%%%%%

\begin{document}

% STUDENT: Your text goes here!
\begin{enumerate}
\item (50\%) Using the word processor or editor of your choice, reproduce, as closely as you can, each of the
following quoted blocks of text. Your version must communicate the same information as my original
and must use symbols and math typesetting where appropriate, but of course, it doesn’t have to be
pixel-for-pixel identical (especially the pseudocode).
\begin{enumerate} 
\item "The \textsc{blort-search} algorithm runs in time $\Theta(n^2)$, while the \textsc{Foo-Search} algorithm runs in time $O(n\log n)$ but is faster in practice for only $n > 10000$.  Do not use \textsc{stupid-search}, which is $\Omega(n^5)$."
\item
"It can be shown that for $\epsilon>0$,
\begin{align*}test
\lim_{n\to\infty}\frac{n\log (n)}{n^{1+\epsilon}}=0
\end{align*}
Moreover,
\begin{align*}
a^{log_b n}=n^{log_b a},
\end{align*}
which is useful for recurrence analysis."
\item "One way to code the \textsc{binary-search} algorithm is as follows:
\begin{algorithmic}
\Procedure{Bsearch}{$x$, $A$, $p$, $r$}
 \If $p=r$
 	\If $A[p]=x$
 		\State \Return {$p$}
 	\Else
 		\State \Return \textit{notFound}
 	\EndIf
 \Else 
 	\State $\textrm{mid} \gets \ceil{(p+r)/2} $
 	\If $A[\textrm{mid}] > x$
 		\State \Return \Call{Bsearch}{$x$, $A$, $p$, $\textrm{mid}-1$}
 	\Else
 		\State \Return \Call{Bsearch}{$x$, $A$, $p$, $r$}
 	\EndIf
 \EndIf
\EndProcedure
\\
This is one of many correct versions of the algorithm."
\end{algorithmic}
\end{enumerate}
\end{enumerate}
\end{document}
