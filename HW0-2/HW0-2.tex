%
% CSE Electronic Homework Template
% Last modified 4/20/2015 by Jeremy Buhler

\documentclass[11pt]{article}
\usepackage[left=0.7in,right=0.7in,top=0.7in,bottom=0.7in]{geometry}
\usepackage{fancyhdr} % for header
\usepackage{graphicx} % for figures
\usepackage{amsmath}  % for extended math markup
\usepackage{amssymb}
\usepackage[bookmarks=false]{hyperref} % for URL embedding
\usepackage[noend]{algpseudocode} % for pseudocode
\usepackage[usenames,dvipsnames]{pstricks} % for drawing
\usepackage{epsfig}
\usepackage{array}

%%%%%%%%%%%%%%%%%%%%%%%%%%%%%%%%%%%%%%%%%%%%%%%%%%%%%%%%%%%%%%%%%%%%%%
% STUDENT: modify the following fields to reflect your
% name/WUSTL Key, the current homework, and the current problem number

% Example: 
%\newcommand{\StudentName}{Jeremy Buhler}
%\newcommand{\WUSTLKey}{jbuhler}

\newcommand{\StudentName}{Alex Bernstein}
\newcommand{\WUSTLKey}{a.bernstein}
\newcommand{\HomeworkNumber}{0}
\newcommand{\ProblemNumber}{2}

%%%%%%%%%%%%%%%%%%%%%%%%%%%%%%%%%%%%%%%%%%%%%%%%%%%%%%%%%%%%%%%%%%%%%%%%
% You can pretty much leave the stuff up to the next line of %%'s alone.

% create header and footer for every page
\pagestyle{fancy}
\fancyhf{}
\lhead{\textbf{\StudentName{} (\WUSTLKey)}}
\chead{\textbf{Homework \HomeworkNumber}}
\rhead{\textbf{Problem \ProblemNumber}}
\cfoot{\thepage}

% preferred pseudocode style
\algrenewcommand{\algorithmicprocedure}{}
\algrenewcommand{\algorithmicthen}{}

% ``do { ... } while (cond)''
\algdef{SE}[DOWHILE]{Do}{doWhile}{\algorithmicdo}[1]{\algorithmicwhile\ #1}%

% ``for (x in y ..bar. z)''
\newcommand{\ForRange}[3]{\For{#1 \textbf{in} #2 \ \ldots \ #3}}

% these are common math formatting commands that aren't defined by default
\newcommand{\union}{\cup}
\newcommand{\isect}{\cap}
\newcommand{\ceil}[1]{\ensuremath \left\lceil #1 \right\rceil}
\newcommand{\floor}[1]{\ensuremath \left\lfloor #1 \right\rfloor}

%%%%%%%%%%%%%%%%%%%%%%%%%%%%%%%%%%%%%%%%%%%%%%%%%%%%%%%%%%%%%%%%%%%%%%

\begin{document}

% STUDENT: Your text goes here!
\begin{enumerate}
\setcounter{enumi}{1}
\item (50\%)
\begin{enumerate}

\item Draw a figure showing a doubly-linked list containing three items, each of which holds an integer
value. Label the head and tail of the list. Then draw an array containing the same three integer
values in the same order.
\\
\\
\begin{center}

\psscalebox{1.0 1.0} % Change this value to rescale the drawing.
{
\begin{pspicture}(0,-6.6)(4.66,6.6)
\pscircle[linecolor=black, linewidth=0.04, dimen=outer](1.4,5.2){1.4}
\pscircle[linecolor=black, linewidth=0.04, dimen=outer](1.4,0.0){1.4}
\pscircle[linecolor=black, linewidth=0.04, dimen=outer](1.4,-5.2){1.4}
\psline[linecolor=black, linewidth=0.04, arrowsize=0.05291666666666667cm 2.0,arrowlength=1.4,arrowinset=0.0]{->}(0.8,3.8)(0.8,1.4)
\psline[linecolor=black, linewidth=0.04, arrowsize=0.05291666666666667cm 2.0,arrowlength=1.4,arrowinset=0.0]{->}(0.8,-1.4)(0.8,-3.8)
\psline[linecolor=black, linewidth=0.04, arrowsize=0.05291666666666667cm 2.0,arrowlength=1.4,arrowinset=0.0]{->}(2.0,-3.8)(2.0,-1.4)
\psline[linecolor=black, linewidth=0.04, arrowsize=0.05291666666666667cm 2.0,arrowlength=1.4,arrowinset=0.0]{->}(2.0,1.4)(2.0,3.8)
\rput[bl](1.2,5.0){\Huge \textbf{6}}
\rput[bl](1.2,-0.2){\Huge \textbf{1}}
\rput[bl](1.2,-5.4){\Huge \textbf{9}}
\rput[bl](2.8,5.0){\huge \textbf{Head}}
\rput[bl](2.8,-5.4){\huge \textbf{Tail}}
\end{pspicture}
}
\end{center}
\item Draw a table comparing the values of the functions $f(x) = x \log x$ (to the nearest integer) and
$g(x) = x^2$

for the following values of x: 1, 10, 100, 1000, 10000, and 100000.
\\
\\
\begin{center}
\begin{tabular}{|l|c|c|r|}
\hline
\textbf{$x$} & \textbf{$f(x)=x \log x$} & \textbf{$g(x)=x^2$}  \\
\hline
$1$          & $0$          & $1$       \\
$10$        & $20$        & $100$        \\
$100$ 		&$500$&$10000$\\
$1000$&$6908$&$100000$\\
$10000$&$92103$&$100000000$\\
$100000$&$1151293$&$1000000000$\\
\hline

\end{tabular}
\end{center}
\end{enumerate}

\end{enumerate}
\end{document}
