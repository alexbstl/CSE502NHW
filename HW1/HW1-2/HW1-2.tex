%
% CSE Electronic Homework Template
% Last modified 4/20/2015 by Jeremy Buhler

\documentclass[11pt]{article}
\usepackage[left=0.7in,right=0.7in,top=0.7in,bottom=0.7in]{geometry}
\usepackage{fancyhdr} % for header
\usepackage{graphicx} % for figures
\usepackage{amsmath}  % for extended math markup
\usepackage{amssymb}
\usepackage[bookmarks=false]{hyperref} % for URL embedding
\usepackage[noend]{algpseudocode} % for pseudocode

%%%%%%%%%%%%%%%%%%%%%%%%%%%%%%%%%%%%%%%%%%%%%%%%%%%%%%%%%%%%%%%%%%%%%%
% STUDENT: modify the following fields to reflect your
% name/WUSTL Key, the current homework, and the current problem number

% Example: 
%\newcommand{\StudentName}{Jeremy Buhler}
%\newcommand{\WUSTLKey}{jbuhler}

\newcommand{\StudentName}{Alex Bernstein}
\newcommand{\WUSTLKey}{a.bernstein}
\newcommand{\HomeworkNumber}{1}
\newcommand{\ProblemNumber}{2}

%%%%%%%%%%%%%%%%%%%%%%%%%%%%%%%%%%%%%%%%%%%%%%%%%%%%%%%%%%%%%%%%%%%%%%%%
% You can pretty much leave the stuff up to the next line of %%'s alone.

% create header and footer for every page
\pagestyle{fancy}
\fancyhf{}
\lhead{\textbf{\StudentName{} (\WUSTLKey)}}
\chead{\textbf{Homework \HomeworkNumber}}
\rhead{\textbf{Problem \ProblemNumber}}
\cfoot{\thepage}

% preferred pseudocode style
\algrenewcommand{\algorithmicprocedure}{}
\algrenewcommand{\algorithmicthen}{}

% ``do { ... } while (cond)''
\algdef{SE}[DOWHILE]{Do}{doWhile}{\algorithmicdo}[1]{\algorithmicwhile\ #1}%

% ``for (x in y ... z)''
\newcommand{\ForRange}[3]{\For{#1 \textbf{in} #2 \ \ldots \ #3}}

% these are common math formatting commands that aren't defined by default
\newcommand{\union}{\cup}
\newcommand{\isect}{\cap}
\newcommand{\ceil}[1]{\ensuremath \left\lceil #1 \right\rceil}
\newcommand{\floor}[1]{\ensuremath \left\lfloor #1 \right\rfloor}

%%%%%%%%%%%%%%%%%%%%%%%%%%%%%%%%%%%%%%%%%%%%%%%%%%%%%%%%%%%%%%%%%%%%%%

\begin{document}

% STUDENT: Your text goes here!
\begin{enumerate}
\setcounter{enumi}{1}

\item (25\%)
\begin{enumerate}
\item 

\begin{algorithmic}
\State $j \gets 1 $ \Comment{1 time}
\While{$ j < n $} \Comment{n-1 times}
    \State $S_1$  \Comment{n-2 times}
	\State $k \gets n$ \Comment{n-2 times}
	\While{$k > j$} \Comment{n-j times}
		\State $S_2$ \Comment{n-j-1 times}
		\State $k--$ \Comment{n-j-1 times}
	\EndWhile
	\State $j++$ \Comment{n-2 times}
\EndWhile
\end{algorithmic}

The Sum of these statments can be repesented by $S(n)$:
\begin{align*}
S(n) &= 1 + (n-1) + 2(n-2) + \sum_{j=1}^{n-1} (n-j+2(n-j-1))
\\&= 3n-4 + \sum_{j=1}^{n-1}(3n-2j-2)
\\&= 3n -4 + 3n(n-2)-2(n-2) -2 \sum_{j=1}^{n-1}j
\\&= 3n -4 + 3n(n-2)-2(n-2) -2(\frac{(n-1)(n)}{2})
\\&= -3n -4 +3n^2 -2n +4 -n^2+n
\\&=2n^2 - 4n
\end{align*}
\item 
\begin{algorithmic}
\State $j \gets 0$ \Comment{1 time}
\While{$j \leq n$} \Comment{n+1 times}
	\State $k \gets 0$ \Comment{n times}
	\While{$k \leq j$} \Comment{j+1 times}
		\State $S_1$ \Comment{j times}
		\State $k++$ \Comment {j times}
	\EndWhile
	\State $j++$ \Comment{n times}
\EndWhile
\end{algorithmic}
Summing these statements:
\begin{align*}
S(n) &= 1 + (n+1) + 2n + \sum_{j=0}^n (2j+j+1)
\\&= 3n+2 + (n+1) + 3 \sum_{j=0}^n j
\\&= 4n+3 + \frac{3n(n+1)}{2}
\\&= 4n + 3 + \frac{3}{2} (n^2+n)
\\&= \frac{11n}{2} + \frac{3n^2}{2} + 3
\end{align*}
\newpage
\item
\begin{algorithmic}
\State $j \gets 1$ \Comment{1 time}
\While{$j \leq n$} \Comment{n times}
	\State $k \gets 1$ \Comment{n-1 times}
	\While{$k \leq j \times j$} \Comment{$\textrm{j}^2$ times}
		\State $S_1$ \Comment{$\textrm{j}^2-1$ times}
		\State $k++$ \Comment {$\textrm{j}^2-1$ times}
	\EndWhile
	\State $j++$ \Comment{n-1 times}
\EndWhile
\end{algorithmic}
Summing these statements:
\begin{align*}
S(n) &= 1+n+n-1+\sum_{j=1}^{n} 3j^2-2
\\&= 2n- 2(n-1) + 3\sum_{j=1}^{n} j^2
\\&= 2 + 3  \frac{n(n+1)(2n+1)}{6} = 2 + \frac{n(n+1)(2n+1)}{2}
\\&= 2+2n^3+3n^2+n
\end{align*}
\end{enumerate}
\end{enumerate}
\end{document}
