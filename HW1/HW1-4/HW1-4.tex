%
% CSE Electronic Homework Template
% Last modified 4/20/2015 by Jeremy Buhler

\documentclass[11pt]{article}
\usepackage[left=0.7in,right=0.7in,top=0.7in,bottom=0.7in]{geometry}
\usepackage{fancyhdr} % for header
\usepackage{graphicx} % for figures
\usepackage{amsmath}  % for extended math markup
\usepackage{amssymb}
\usepackage{amsthm}
\usepackage[bookmarks=false]{hyperref} % for URL embedding
\usepackage[noend]{algpseudocode} % for pseudocode

%%%%%%%%%%%%%%%%%%%%%%%%%%%%%%%%%%%%%%%%%%%%%%%%%%%%%%%%%%%%%%%%%%%%%%
% STUDENT: modify the following fields to reflect your
% name/WUSTL Key, the current homework, and the current problem number

% Example: 
%\newcommand{\StudentName}{Jeremy Buhler}
%\newcommand{\WUSTLKey}{jbuhler}

\newcommand{\StudentName}{Alex Bernstein}
\newcommand{\WUSTLKey}{a.bernstein}
\newcommand{\HomeworkNumber}{1}
\newcommand{\ProblemNumber}{1}

%%%%%%%%%%%%%%%%%%%%%%%%%%%%%%%%%%%%%%%%%%%%%%%%%%%%%%%%%%%%%%%%%%%%%%%%
% You can pretty much leave the stuff up to the next line of %%'s alone.

% create header and footer for every page
\pagestyle{fancy}
\fancyhf{}
\lhead{\textbf{\StudentName{} (\WUSTLKey)}}
\chead{\textbf{Homework \HomeworkNumber}}
\rhead{\textbf{Problem \ProblemNumber}}
\cfoot{\thepage}

% preferred pseudocode style
\algrenewcommand{\algorithmicprocedure}{}
\algrenewcommand{\algorithmicthen}{}

% ``do { ... } while (cond)''
\algdef{SE}[DOWHILE]{Do}{doWhile}{\algorithmicdo}[1]{\algorithmicwhile\ #1}%

% ``for (x in y ... z)''
\newcommand{\ForRange}[3]{\For{#1 \textbf{in} #2 \ \ldots \ #3}}

% these are common math formatting commands that aren't defined by default
\newcommand{\union}{\cup}
\newcommand{\isect}{\cap}
\newcommand{\ceil}[1]{\ensuremath \left\lceil #1 \right\rceil}
\newcommand{\floor}[1]{\ensuremath \left\lfloor #1 \right\rfloor}

%%%%%%%%%%%%%%%%%%%%%%%%%%%%%%%%%%%%%%%%%%%%%%%%%%%%%%%%%%%%%%%%%%%%%%

\begin{document}

% STUDENT: Your text goes here!
\begin{enumerate}
\setcounter{enumi}{3}
\item (25\%)
\begin{enumerate}
\item Does $(n+1)^2 = \Omega(n \log n)$?
\\
True.
\begin{proof}
Let $f(n) = (n+1)^2$ and $g(n)= n \log n$
Take 
\begin{align*}
\lim_{n \to \infty}\frac{f(n)}{g(n)} &= \lim_{n \to \infty} \frac{(n+1)^2}{n \log n}
\\&= \lim_{n \to \infty} \frac{n^2+2n+1}{n \log n} 
\\&= + \infty
\end{align*}
So, by the limit comparison test, $(n+1)^2 = \Omega(n \log n)$.
\end{proof}
\item Does $3^{9 \log n + \log \log n} = \Omega{(n^3)}$?
\\
True.
\begin{proof}
Note that:
\begin{align*}
f(n) = 3^{9 \log n + \log \log n} &= 3^{9 \log n }3^{\log \log n}
\\&=3^{(\log n^9)} 3^{\log \log n}
\\&=(n^9)^{\log(3)} 3^{\log \log n}
\\&=n^{9 \log(3)} 3^{\log \log n}
\end{align*}
Also note both that $\log_2(\log_2 (n)) \geq 1 \forall n\geq 4$, and $9(\log_2(3)) \approx 14$ (it's actually slightly greater).  Therefore,
$f(n)$ can be bounded below by $3n^{14}$  $\forall n \geq 4$.  Therefore:
\begin{align*}
\lim_{n \to \infty} \frac{3^{9 \log n + \log \log n}} {n^3} \geq \lim_{n \to \infty} \frac{3n^{14}}{n^3} = +\infty
\end{align*} 
\end{proof}
\item Does $n \log_5 n = \Theta(n \ln n)$?
\\
True.
\begin{proof}
Note that:
\begin{align*}
n \log_5{5} = n \frac{\ln{n}}{\ln{5}}
\end{align*}
and that:
\begin{align*}
\lim_{n \to \infty} \frac{n \log_5{n}}{ n \ln{n}} = \lim_{n \to \infty}\frac{1}{\ln{5}}= \frac{1}{\ln{5}}
\end{align*}
Because this limit is a constant, $n \log_5{n} = \Theta(n \ln{n})$
\end{proof}
\newpage
\item Does $(n-2)^2 = \Theta(n \log n)$?
\\
False.
\begin{proof}
Note that:
\begin{align*}
\lim_{n \to \infty}\frac{(n-2)^2}{n \log{n}} = \lim_{n \to \infty} \frac{n^2-4n+4}{n \log{n}} &= + \infty 
\\&\neq c \textrm{  },\textrm{  } c < \infty
\end{align*}
Therefore, $(n-2)^2 \neq \Theta(n \log{n})$.
\end{proof}
\item Does $n^{\frac{61}{60}}= \mathcal{O}(n \log{n})$?
\\
False.
\begin{proof}
\begin{align*}
\lim_{n \to \infty} \frac{n^{\frac{61}{60}}}{n \log n}= + \infty
\end{align*}
Therefore, $n^{\frac{61}{60}}$ is not $\mathcal{O}(n \log{n})$ but it is $\Omega(n \log{n})$.
\end{proof}
\item Let $f(n)$ and $g(n)$ be non-negative functions of $n$.  If $f(n)=\mathcal{O}(g(n))$, does $f(n)+g(n)= \Theta(g(n))$?
\\
True.
\begin{proof}
Because $f(n)=\mathcal{O}(g(n))$:
\begin{align*}
\lim_{n \to \infty} \frac{f(n)}{g(n)} = 0.
\end{align*}
Therefore, we have:
\begin{align*}
\lim_{n \to \infty} \frac{f(n)+g(n)}{g(n)} &= \lim_{n \to \infty} \frac{f(n)}{g(n)} + \frac{g(n)}{g(n)}=1
\\&=\lim_{n \to \infty} 0 + \lim_{n \to \infty} \frac{g(n)}{g(n)} = 1 < \infty.
\end{align*}
Therefore, $f(n)+g(n)= \Theta(g(n))$.
\end{proof}
\item Let $f(n)$ and $g(n)$ be non-negative functions of $n$.  If $f(n)=\Theta(g(n))$, does $\frac{f(n)}{g(n)}= \Theta(1)$?
\\
True.
\begin{proof}
By the provided relation, we know:
\begin{align*}
\lim_{n \to \infty}\frac{f(n)}{g(n)}=c< \infty.
\end{align*}
Therefore, it is trivial to show that:
\begin{align*}
\lim_{n \to \infty}\frac{\Bigl(\frac{f(n)}{g(n)}\Bigr)}{1}=\lim_{n \to \infty}\frac{f(n)}{g(n)} = c< \infty.
\end{align*}
Therefore, $\frac{f(n)}{g(n)} = \Theta(1)$.
\end{proof}
\end{enumerate}
\end{enumerate}
\end{document}
