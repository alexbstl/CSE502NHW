%
% CSE Electronic Homework Template
% Last modified 4/20/2015 by Jeremy Buhler

\documentclass[11pt]{article}
\usepackage[left=0.7in,right=0.7in,top=0.7in,bottom=0.7in]{geometry}
\usepackage{fancyhdr} % for header
\usepackage{graphicx} % for figures
\usepackage{amsmath}  % for extended math markup
\usepackage{amssymb}
\usepackage[bookmarks=false]{hyperref} % for URL embedding
\usepackage[noend]{algpseudocode} % for pseudocode
\usepackage{qtree}

%%%%%%%%%%%%%%%%%%%%%%%%%%%%%%%%%%%%%%%%%%%%%%%%%%%%%%%%%%%%%%%%%%%%%%
% STUDENT: modify the following fields to reflect your
% name/WUSTL Key, the current homework, and the current problem number

% Example: 
%\newcommand{\StudentName}{Jeremy Buhler}
%\newcommand{\WUSTLKey}{jbuhler}

\newcommand{\StudentName}{Alex Bernstein}
\newcommand{\WUSTLKey}{a.bernstein}
\newcommand{\HomeworkNumber}{1}
\newcommand{\ProblemNumber}{3}

%%%%%%%%%%%%%%%%%%%%%%%%%%%%%%%%%%%%%%%%%%%%%%%%%%%%%%%%%%%%%%%%%%%%%%%%
% You can pretty much leave the stuff up to the next line of %%'s alone.

% create header and footer for every page
\pagestyle{fancy}
\fancyhf{}
\lhead{\textbf{\StudentName{} (\WUSTLKey)}}
\chead{\textbf{Homework \HomeworkNumber}}
\rhead{\textbf{Problem \ProblemNumber}}
\cfoot{\thepage}

% preferred pseudocode style
\algrenewcommand{\algorithmicprocedure}{}
\algrenewcommand{\algorithmicthen}{}

% ``do { ... } while (cond)''
\algdef{SE}[DOWHILE]{Do}{doWhile}{\algorithmicdo}[1]{\algorithmicwhile\ #1}%

% ``for (x in y ... z)''
\newcommand{\ForRange}[3]{\For{#1 \textbf{in} #2 \ \ldots \ #3}}

% these are common math formatting commands that aren't defined by default
\newcommand{\union}{\cup}
\newcommand{\isect}{\cap}
\newcommand{\ceil}[1]{\ensuremath \left\lceil #1 \right\rceil}
\newcommand{\floor}[1]{\ensuremath \left\lfloor #1 \right\rfloor}

%%%%%%%%%%%%%%%%%%%%%%%%%%%%%%%%%%%%%%%%%%%%%%%%%%%%%%%%%%%%%%%%%%%%%%

\begin{document}

% STUDENT: Your text goes here!
\begin{enumerate}
\setcounter{enumi}{2}
\item (25\%)

\begin{enumerate}
\item The recurrence relation for $T(n)$ of the modified algorithm is: 
\begin{align*}
T(n) = 
	\begin{cases}
	c_0 & \text{if } x = 1 \\
	5T(\frac{n}{5}) + cn & \text{otherwise}
	\end{cases}
\end{align*}
This is the same as:
\begin{align*}
T(n) &= c_0n+ 5\sum_{k = 0}^{\log_5(\frac{n}{5})-1}cn
\end{align*}
\item Recurrence Tree:
\noindent

\begin{center}
\Tree [.O O\\.\\.\\.\\$c_0$ O\\.\\.\\.\\$c_0$ [.O O\\.\\.\\.\\$c_0$ O\\.\\.\\.\\$c_0$ O\\.\\.\\.\\$c_0$ O\\.\\.\\.\\$c_0$ O\\.\\.\\.\\$c_0$ ] O\\.\\.\\.\\$c_0$ O\\recurrence.\\.\\.\\$c_0$ ]
\end{center}
where the work per node is $\frac{cn}{5^k}$ where $k$ is the level of the node.  The work per level can then be generalized to $5^k (\frac{cn}{5^k})=cn$.


\item We wish to show that $T(n) = \Theta(n \log (n))$
\begin{align*}
T(n) &= c_0n+ \sum_{k = 0}^{\log_5(\frac{n}{5})-1}cn \\&= n(c_0 + c\log_5 \Bigl(\frac{n}{5}\Bigr))
\\&= n(c_0+c\log_5(n) - \log_5(5) )
\\&= n(c_0+c\log_5(n) - 1 )
\\&= n(c_1+c\log_5(n))
\\&= n(c_1+c\Bigl(\frac{\log_2(n)}{\log_2(5)} \Bigr))
\\&= n(c_1+c_2\log_2(n))
\\&= nc_1 + c_3n\log_2(n)
\\&= \Theta ( n \log(n))
\end{align*}
\item By dividing the problem into $m$ subproblems, we divide $n$ into $m$ pieces at each level, and repeat each subproblem $m$ times.  This results in a different logarithmic base and leading coefficient.  Because of the change of base, this changes the constant coefficient multiplier, which doesn't alter the asymptotic result.
\end{enumerate}
\end{enumerate}
\end{document}
