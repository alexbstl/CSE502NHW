%
% CSE Electronic Homework Template
% Last modified 4/20/2015 by Jeremy Buhler

\documentclass[11pt]{article}
\usepackage[left=0.7in,right=0.7in,top=0.7in,bottom=0.7in]{geometry}
\usepackage{fancyhdr} % for header
\usepackage{graphicx} % for figures
\usepackage{amsmath}  % for extended math markup
\usepackage{amssymb}
\usepackage[bookmarks=false]{hyperref} % for URL embedding
\usepackage[noend]{algpseudocode} % for pseudocode

%%%%%%%%%%%%%%%%%%%%%%%%%%%%%%%%%%%%%%%%%%%%%%%%%%%%%%%%%%%%%%%%%%%%%%
% STUDENT: modify the following fields to reflect your
% name/WUSTL Key, the current homework, and the current problem number

% Example: 
%\newcommand{\StudentName}{Jeremy Buhler}
%\newcommand{\WUSTLKey}{jbuhler}

\newcommand{\StudentName}{Alex Bernstein}
\newcommand{\WUSTLKey}{a.bernstein}
\newcommand{\HomeworkNumber}{1}
\newcommand{\ProblemNumber}{1}

%%%%%%%%%%%%%%%%%%%%%%%%%%%%%%%%%%%%%%%%%%%%%%%%%%%%%%%%%%%%%%%%%%%%%%%%
% You can pretty much leave the stuff up to the next line of %%'s alone.

% create header and footer for every page
\pagestyle{fancy}
\fancyhf{}
\lhead{\textbf{\StudentName{} (\WUSTLKey)}}
\chead{\textbf{Homework \HomeworkNumber}}
\rhead{\textbf{Problem \ProblemNumber}}
\cfoot{\thepage}

% preferred pseudocode style
\algrenewcommand{\algorithmicprocedure}{}
\algrenewcommand{\algorithmicthen}{}

% ``do { ... } while (cond)''
\algdef{SE}[DOWHILE]{Do}{doWhile}{\algorithmicdo}[1]{\algorithmicwhile\ #1}%

% ``for (x in y ... z)''
\newcommand{\ForRange}[3]{\For{#1 \textbf{in} #2 \ \ldots \ #3}}

% these are common math formatting commands that aren't defined by default
\newcommand{\union}{\cup}
\newcommand{\isect}{\cap}
\newcommand{\ceil}[1]{\ensuremath \left\lceil #1 \right\rceil}
\newcommand{\floor}[1]{\ensuremath \left\lfloor #1 \right\rfloor}

%%%%%%%%%%%%%%%%%%%%%%%%%%%%%%%%%%%%%%%%%%%%%%%%%%%%%%%%%%%%%%%%%%%%%%

\begin{document}

% STUDENT: Your text goes here!
\begin{enumerate}
\item (25\%)
\begin{enumerate}
\item What is the smallest problem size $n_0$ such that algorithm $\mathit{B}$ is (strictly) faster than algorithm $\mathit{A}$
for all $n \geq n_0$?
\\
Setting $0.03n^2 = 0.15n \log n+0.00001n^2$, we find the these two functions intersect at $n=1.1772$ and $n=22.451$, and $0.03n^2$ is smaller over that interval.  Because $n=1$ is a trivial case, this means that the smallest number $n$ such that running time of $\mathit{B}=0.15n \log n+0.00001n^2$ seconds is faster than the running time of  $\mathit{A}= 0.03n^2$ seconds is at $n=23$ points.
\\
\item What is the smallest problem size $n_1$ such that algorithm $\mathit{C}$ is (strictly) faster than algorithm $\mathit{B}$ for all $n \geq n_1$?
\\
Setting the running time of $\mathit{B}$ equal to the running time of $\mathit{C}$, we must solve the following equation for $n$:
\begin{align*}
0.15n \log n > n
\end{align*}
Simple algebra results in the relation:
\begin{align*}
n > 2^{\frac{1}{.15}}= 2 ^{\frac{20}{3}} = 101.549
\end{align*}
So $\mathit{C}$ is strictly faster that $\mathit{B}$ for $n \geq 102$.
\item Describe how to construct a distance computing algorithm that always achieves the best running time of any algorithm $\mathit{A}, \mathit{B} \textrm{, and } \mathit{C}$ on it's input.
\\
This would simply be to create an algorithm that runs either $\mathit{A}, \mathit{B} \textrm{, or } \mathit{C}$ based on input size.  So, we can create algorithm $D$ that runs either  $\mathit{A}, \mathit{B} \textrm{, or } \mathit{C}$ based on $n$, the size of the input:
\begin{align*}
D = 	\begin{cases}
		A & \text{if } n < 23 \\
		B & \text{if } 102>n\geq 23 \\
		C & \text{if } n\geq 102
	\end{cases}
\end{align*}
This would always result in the fastest possible output of all three algorithms.
\\
\item Professor Nikrasch suggests processing the Lotto results using a different clustering algorithm
altogether, one which avoids computing distances between results. This new algorithm runs in $n^{1.2}$ seconds of an input of size $n$. Is this algorithm ever faster than the fastest of $\mathit{A}, \mathit{B} \textrm{, or } \mathit{C}$?  If so, for what value of $n$ does it start to win?
\\
In order to check if this new algorithm (call it $\mathit{E}$ for simplicity) is ever faster than the fastest of  $\mathit{A}, \mathit{B} \textrm{, or } \mathit{C}$, we can solve for $n$ in the following relation (where each character represents the runtime of the algorithm in terms of input size $n$):
\begin{align*}
E < D.
\end{align*}
Solving for this, we find that $n = 1.65235 \times 10^6$, or $1,652,350$ points.
\end{enumerate}
\end{enumerate}
\end{document}
