%
% CSE Electronic Homework Template
% Last modified 4/20/2015 by Jeremy Buhler

\documentclass[11pt]{article}
\usepackage[left=0.7in,right=0.7in,top=0.7in,bottom=0.7in]{geometry}
\usepackage{fancyhdr} % for header
\usepackage{graphicx} % for figures
\usepackage{amsmath}  % for extended math markup
\usepackage{amsthm}
\usepackage{amssymb}
\usepackage[bookmarks=false]{hyperref} % for URL embedding
\usepackage[noend]{algpseudocode} % for pseudocode

%%%%%%%%%%%%%%%%%%%%%%%%%%%%%%%%%%%%%%%%%%%%%%%%%%%%%%%%%%%%%%%%%%%%%%
% STUDENT: modify the following fields to reflect your
% name/WUSTL Key, the current homework, and the current problem number

% Example: 
%\newcommand{\StudentName}{Jeremy Buhler}
%\newcommand{\WUSTLKey}{jbuhler}

\newcommand{\StudentName}{Alex Bernstein}
\newcommand{\WUSTLKey}{a.bernstein}
\newcommand{\HomeworkNumber}{1}
\newcommand{\ProblemNumber}{1}

%%%%%%%%%%%%%%%%%%%%%%%%%%%%%%%%%%%%%%%%%%%%%%%%%%%%%%%%%%%%%%%%%%%%%%%%
% You can pretty much leave the stuff up to the next line of %%'s alone.

% create header and footer for every page
\pagestyle{fancy}
\fancyhf{}
\lhead{\textbf{\StudentName{} (\WUSTLKey)}}
\chead{\textbf{Homework \HomeworkNumber}}
\rhead{\textbf{Problem \ProblemNumber}}
\cfoot{\thepage}

% preferred pseudocode style
\algrenewcommand{\algorithmicprocedure}{}
\algrenewcommand{\algorithmicthen}{}

% ``do { ... } while (cond)''
\algdef{SE}[DOWHILE]{Do}{doWhile}{\algorithmicdo}[1]{\algorithmicwhile\ #1}%

% ``for (x in y ... z)''
\newcommand{\ForRange}[3]{\For{#1 \textbf{in} #2 \ \ldots \ #3}}

% these are common math formatting commands that aren't defined by default
\newcommand{\union}{\cup}
\newcommand{\isect}{\cap}
\newcommand{\ceil}[1]{\ensuremath \left\lceil #1 \right\rceil}
\newcommand{\floor}[1]{\ensuremath \left\lfloor #1 \right\rfloor}

%%%%%%%%%%%%%%%%%%%%%%%%%%%%%%%%%%%%%%%%%%%%%%%%%%%%%%%%%%%%%%%%%%%%%%

\begin{document}

% STUDENT: Your text goes here!
\begin{enumerate}
\setcounter{enumi}{3}
\item (25\%)
This problem is an example of \textit{universal hashing},
a strategy for picking hash functions for a hash table randomly so
that no input always exhibits bad hashing behavior.

Let $p$ be a prime number.  I want to hash pairs of numbers $(x,y)$,
where $x$ and $y$ are always between 0 and $p-1$ inclusive.  I
decide to use a chained hash table with hash function
\[
h_{a,b}(x,y) = (ax + by) \bmod p
\]
where $a$ and $b$ also lie between 0 and $p-1$.
\begin{enumerate}
\item Suppose that my $a$ and $b$ are fixed, and that you've discovered what
they are (perhaps by hacking into my computer).  Describe how to
generate $p$ distinct input pairs $(x_i, y_i)$ for which
$h_{a,b}(x_i,y_i)$ yields the same value.  That is, all the inputs
$(x_i,y_i)$ will hash to the same slot of my table.

\textit{(\textbf{Hint}: for any $c$, $1 \leq c < p$, and every $i$, $0
\leq i < p$, there exists exactly one $j$, $0 \leq j < p$ for which
$cj \equiv i \pmod p$.)}
\begin{proof}
First, note that for any given $p,z \in \{0,1,2,\ldots,p-1\}= \mathbb{Z}_p$, we have a unique choice of $q \in \mathbb{Z}_p$ such that $p+q=z$.  Further, we know that for any given $p$, the $q$ described above always exists.  That is, to any number in $\mathbb{Z}_p$, we can always add another, unique number to construct any other number in $\mathbb{Z}_p$.  Therefore, because we want to construct a series of pairs of values $(x_1,y_1),(x_2,y_2)\ldots(x_p,y_p)$ such that they all hash to the same value, we must first choose some value in $\{0,1,2,\ldots,p-1\} = \mathbb{Z}_p$ that we want every value to hash to.  Given that sum value, say $z$, we can allow $x_i$ to vary over each value in $\mathbb{Z}_p$, in total, $p$ values.  Given that we know both $a,b$, we know the value for each $ax_i$, so we wish to construct $by_i$ such that $by_i= z - ax_i \bmod p$.  Given that this $by_i$ is unique, and $b$ is already fixed, we can simply choose the unique $y_i$ as described in the hint such that $by_i \bmod p = (z - ax_i) \bmod p$.  Therefore, we have now constructed $p$ different pairs $(x_i,y_i)$ such that $(ax_i+by_i)\bmod p = z$, $i = 1,2,\ldots p$.
\end{proof}



\item To defend against your malicious hackery, I have decided not to fix
$a$ and $b$ once and for all, but rather to choose them randomly every
time I instantiate my hash table class.  Each value will be chosen
uniformly at random (with replacement) from the range $0 \ldots p-1$.

Fix two non-identical inputs $(x,y)$ and $(x',y')$.  (They may have
the same $x$ or $y$ values, but not both.) For how many distinct pairs
$(a,b)$ will these two inputs hash to the same slot?

(Use the hint from part (a)).

\begin{proof}
Given any $(x,y)$ and $(x',y')$ we wish to find the number of pairs $(a,b)$ such that:
\begin{align*}
(ax+by)\bmod p = (ax'+by')\bmod p \text{ and } \\
(ax+by)\bmod p - (ax'+by')\bmod p = 0 
 \end{align*}
 given that $a,b,x,y,x',y' \in \mathbb{Z}_p = \{0,1,2,\ldots,p-1\}$.  This is the same as the following expression:
\begin{align*}
 (ax+by)-(ax'+by') = mp
\end{align*}
where $m \in \mathbb{Z}$ because $mp$ is congruent to $0 \bmod p$ $\forall m \in \mathbb{Z}$.
Therefore, we can say:
\begin{align*}
a(x-x')+b(y-y')  &= mp
\\&=0 \bmod p
\end{align*}  
Based on the initial assumptions, we are left with 2 cases:
\begin{enumerate}
\item $x=x'$ or $y=y'$.
\\
Without lose of generality, assume that $x=x'$ and $y \neq y'$.  In this case, from the previous expression, we have:
\begin{align*}
a(0) + b(y-y')= 0 \bmod p
\end{align*}
Because $a(0) = 0$, we must have that:$
b(y-y') = 0$.  $y-y'\neq 0$ is fixed, therefore, by the hint in part \textbf{(a)}, we must choose the unique $b$ such that $b(y-y')=0$.  Therefore, there can only be a single choice for $b$, but $p$ choices for $a$, because $a$ is trivial.  Therefore, there are trivially $p$ choices for $(a,b)$, as $a$ varies over all of $\mathbb{Z}_p$ and $b$ is fixed.
\\
\item $x \neq x'$ and $y \neq y'$.

Let $x-x'=q$ and $y-y'=r$.  In this case, we have:
\begin{align*}
a(x-x')+b(y-y') =& \\
a(q)+b(r)=& 0 \bmod p
\end{align*}
By the hint in \textbf{(a)}, we know that because $q,r \neq 0$ are fixed, the equation $aq = z\bmod p$ has exactly one solution for every element $z\in \mathbb{Z}_p$.  Similarly, for every $z \in \mathbb{Z}_p$, there exists exactly one $-z \in \mathbb{Z}_p$ such that $-z + z =0$.  Therefore, as $a$ is allowed to vary, over the entire set, $b$ must be chosen uniquely such that $aq+br = 0 \bmod p$ holds.  Because there are $p$ possible choices for $a$ (i.e. every element in $\mathbb{Z}_p$), there are $p$ unique combinations $(a,b)$ such that $(ax+by) \bmod p = (ax'+by') \bmod p$. 
\end{enumerate}
\end{proof}
\item
If I choose each of $a$ and $b$ uniformly at random from $0 \ldots p-1$,
what is the probability that $(x,y)$ and $(x',y')$ will hash to the same value?  
\\
We have a $\frac{1}{p}$ chance that $(x,y)$ and $(x',y')$ will hash to the same value.
\begin{proof}
Notice that for $\{0,1,2,\ldots,p-1\}$ there are $p^2$ ordered pairs $(a,b)$ that can be formed.  From \textbf{(b)}, we know that there are $p$ unique pairs $(a,b)$ such that $(ax+by)\bmod p = (ax'+by') \bmod p$.  Therefore, the probability that we choose $a,b$ randomly from $\{0,1,2,\ldots,p-1\}$ such that $(x,y)$ and $(x',y')$ hash to the same values is given by:
\begin{align*}
P[(ax+by)\bmod p = (ax'+by')\bmod p] = \frac{p}{p^2} = \frac{1}{p}.
\end{align*}
\end{proof}
\item
Given an \emph{arbitrary} set of $n$ distinct inputs $(x_i, y_i)$,
what is the expected number (over my random choices of $a$ and $b$)
of pairs $i, j$, $i < j$, for which $h_{a,b}(x_i, y_i) = h_{a,b}(x_j,y_j)$?

\textit{(\textbf{Hint}: use linearity of expectation!)}

\begin{proof}
We know from \textbf{(c)} that the probability of two pairs colliding is $\frac{1}{p}$.  Using the indicator function 
\begin{align*}
\chi_{si} = 
\begin{cases}
1 \text{ if key } i \text{ hashes to slot } s \\
0 \text{  otherwise} 
\end{cases}
\end{align*}

The chance that two points out of $n$ will hash to the same slot is given by 
$n \choose 2$ with a probability of $\frac{1}{p}$ per slot.  
Therefore, the chance that 2 pairs will collide in any given slot is given by 
\begin{align*}
\frac{\binom{n}{2}}{p}
\end{align*}
\end{proof}
\end{enumerate}
\end{enumerate}
\end{document}
