%
% CSE Electronic Homework Template
% Last modified 4/20/2015 by Jeremy Buhler

\documentclass[11pt]{article}
\usepackage[left=0.7in,right=0.7in,top=0.7in,bottom=0.7in]{geometry}
\usepackage{fancyhdr} % for header
\usepackage{graphicx} % for figures
\usepackage{amsmath}  % for extended math markup
\usepackage{amssymb}
\usepackage{amsthm}
\usepackage[bookmarks=false]{hyperref} % for URL embedding
\usepackage[noend]{algpseudocode} % for pseudocode

%%%%%%%%%%%%%%%%%%%%%%%%%%%%%%%%%%%%%%%%%%%%%%%%%%%%%%%%%%%%%%%%%%%%%%
% STUDENT: modify the following fields to reflect your
% name/WUSTL Key, the current homework, and the current problem number

% Example: 
%\newcommand{\StudentName}{Jeremy Buhler}
%\newcommand{\WUSTLKey}{jbuhler}

\newcommand{\StudentName}{Alex Bernstein}
\newcommand{\WUSTLKey}{a.bernstein}
\newcommand{\HomeworkNumber}{2}
\newcommand{\ProblemNumber}{1}

%%%%%%%%%%%%%%%%%%%%%%%%%%%%%%%%%%%%%%%%%%%%%%%%%%%%%%%%%%%%%%%%%%%%%%%%
% You can pretty much leave the stuff up to the next line of %%'s alone.

% create header and footer for every page
\pagestyle{fancy}
\fancyhf{}
\lhead{\textbf{\StudentName{} (\WUSTLKey)}}
\chead{\textbf{Homework \HomeworkNumber}}
\rhead{\textbf{Problem \ProblemNumber}}
\cfoot{\thepage}

% preferred pseudocode style
\algrenewcommand{\algorithmicprocedure}{}
\algrenewcommand{\algorithmicthen}{}

% ``do { ... } while (cond)''
\algdef{SE}[DOWHILE]{Do}{doWhile}{\algorithmicdo}[1]{\algorithmicwhile\ #1}%

% ``for (x in y ... z)''
\newcommand{\ForRange}[3]{\For{#1 \textbf{in} #2 \ \ldots \ #3}}

% these are common math formatting commands that aren't defined by default
\newcommand{\union}{\cup}
\newcommand{\isect}{\cap}
\newcommand{\ceil}[1]{\ensuremath \left\lceil #1 \right\rceil}
\newcommand{\floor}[1]{\ensuremath \left\lfloor #1 \right\rfloor}

%%%%%%%%%%%%%%%%%%%%%%%%%%%%%%%%%%%%%%%%%%%%%%%%%%%%%%%%%%%%%%%%%%%%%%

\begin{document}

% STUDENT: Your text goes here!
\begin{enumerate}
\item (25\%)  What Asymptotic solution, if any, does the master method give  for each of the following recurrences?
\\
\begin{enumerate}
\item $T(n) = 16T(\frac{n}{2}) + 3n^4\log^2(n) = \Theta(n^4 \log^2(n))$.
\begin{proof}
Let $a = 16$, $b = 2$, $f(n) = 3n^4 \log^2(n)$.  Clearly, 
\begin{align*}
\log_2(16) = 4 \text{ and } \\
f(n) = \Theta(n^4 \log_2^2(n)).
\end{align*}  Therefore, by the Master Method Theorem, 
\begin{align*}
T(n) = \Theta(n^4 \log_2^3(n)).
\end{align*}
\end{proof}
\item $T(n) = T(\frac{n}{3}) + 4 \log(n) = \Theta( \log^2(n))$.
\begin{proof}
Let $a = 1$, $b = 3$, and $f(n) = 4 \log(n)$.  Clearly,
\begin{align*}
\log_3(1) = 0.
\end{align*}
And, we can say that:
\begin{align*}
f(n) = \Theta(n^0 \log^1(n))= \Theta( \log(n)).
\end{align*}
Therefore,
\begin{align*}
T(n) = \Theta(\log^2(n)).
\end{align*}
\end{proof}
\item $T(n) = 4T(\frac{n}{4}) + \frac{n}{\log \log n}$ cannot be judged by the Master Method.
\begin{proof}
Let $a = 4$, $b=4$, and $f(n) = \frac{n}{\log \log n}$.  Clearly, $\log_b(a) = \log_4(4) = 1$.  \\
Also, comparing $g(n)=n$ to $f(n)=\frac{n}{\log \log n}{n}$ with the limit comparison test, 
\begin{align*}
\lim_{n \to \infty}\frac{f(n)}{g(n)}=\lim_{n \to \infty} \frac{\frac{n}{\log \log n}}{n} = \lim_{n \to \infty} \frac{1}{\log \log n} = 0.
\end{align*}
Therefore, we can say that
\begin{align*}
f(n) = \frac{n}{\log \log n}  =\mathcal{O}(n).
\end{align*}
However, for any $\epsilon \in \mathbb{R^+}$ (i.e. $\epsilon>0$),
\begin{align*}
\lim_{n \to \infty} \frac{f(n)}{g(n)} = \lim_{n \to \infty} \frac{\frac{n}{\log \log n}}{n^{1-\epsilon}}= \lim_{n \to \infty} \frac{n^{\epsilon}}{\log \log n} = \infty.
\end{align*}
So, we can say that $f(n) = \Omega(n^{1-\epsilon}) \text{ } \forall \epsilon>0$.  For the same conditions on $\epsilon$, we can say that $f(n) = \mathcal{O}(n^{1+\epsilon})$.
Therefore we cannot extract any useful information about this recurrence from the Master Method.
\end{proof}
\item $T(n) =  9 T\left(\frac{n}{4}\right) + 2n^{\log_4 9}= \Theta \left(n^{\log_4(9)}\log(n)\right)$.
\begin{proof}
Let $a = 9$, $b=4$, and $f(n) = 2n^{\log_4(9)}$.  It is trivial to show that, $f(n) = \Theta\left(n^{\log_4(9)}\right) = \Theta\left(n^{\log_4(9)}\log^0(n)\right)$.  Therefore, via the Master Method, we can say that $T(n)=\Theta\left(n^{\log_4(9)} \log(n) \right)$. 
\end{proof}
\item $T(n) =  11 T\left(\frac{n}{3}\right) + n^2 = \Theta\left(n^{\log_3(11)}\right)$.
\begin{proof}
Let $a = 11$, $b=3$, and $f(n) = n^2$.  Quick inspection tells us that $\log_3(11)>2$, so we can say that $f(n) = \mathcal{O}\left(n^{\log_3(11)-\epsilon}\right)$ for a small $\epsilon>0$. We can therefore say that $T(n) = \Theta\left(n^{\log_3(11)}\right)$ by the Master Method.
\end{proof}
\item $T(n) = 999 T\left(\frac{n}{10}\right) + 2n^3 = \Theta(n^3)$.
\begin{proof}
Let $a = 999$, $b = 10$, and $f(n) = 2n^3$.  Quick inspection tells us that 
$\log_10(999)<3$, so $f(n) = 2n^3 = \Omega(n^{\log_{10}(999)+ \epsilon})$ for a sufficent $\epsilon>0$.  Therefore, we can say that $T(n) = \Theta(n^3)$ by the Master Method.
\end{proof}
\item $T(n) =  7 T\left(\frac{n}{2}\right) + n^3 \log n \log \log n = \Theta(n^3 \log n \log \log n)$.
\begin{proof}
Let $a= 7$, $b=2$, and $f(n) = n^3 \log n \log \log n$.  Quick inspection tells us that $\log_2(7) < 3$.  We can therefore say that $f(n)= \Omega(n^{\log_2(7)+ \epsilon})$ for a sufficient $\epsilon>0$.  Therefore, by the Master Method, $T(n) = \Theta(n^3 \log n \log \log n)$.
\end{proof}

\end{enumerate}
\end{enumerate}
\end{document}
