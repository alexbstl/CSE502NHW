%
% CSE Electronic Homework Template
% Last modified 4/20/2015 by Jeremy Buhler

\documentclass[11pt]{article}
\usepackage[left=0.7in,right=0.7in,top=0.7in,bottom=0.7in]{geometry}
\usepackage{fancyhdr} % for header
\usepackage{graphicx} % for figures
\usepackage{amsmath}  % for extended math markup
\usepackage{amsthm}
\usepackage{amssymb}
\usepackage[bookmarks=false]{hyperref} % for URL embedding
\usepackage[noend]{algpseudocode} % for pseudocode

%%%%%%%%%%%%%%%%%%%%%%%%%%%%%%%%%%%%%%%%%%%%%%%%%%%%%%%%%%%%%%%%%%%%%%
% STUDENT: modify the following fields to reflect your
% name/WUSTL Key, the current homework, and the current problem number

% Example: 
%\newcommand{\StudentName}{Jeremy Buhler}
%\newcommand{\WUSTLKey}{jbuhler}

\newcommand{\StudentName}{Alex Bernstein}
\newcommand{\WUSTLKey}{a.bernstein}
\newcommand{\HomeworkNumber}{2}
\newcommand{\ProblemNumber}{1}

%%%%%%%%%%%%%%%%%%%%%%%%%%%%%%%%%%%%%%%%%%%%%%%%%%%%%%%%%%%%%%%%%%%%%%%%
% You can pretty much leave the stuff up to the next line of %%'s alone.

% create header and footer for every page
\pagestyle{fancy}
\fancyhf{}
\lhead{\textbf{\StudentName{} (\WUSTLKey)}}
\chead{\textbf{Homework \HomeworkNumber}}
\rhead{\textbf{Problem \ProblemNumber}}
\cfoot{\thepage}

% preferred pseudocode style
\algrenewcommand{\algorithmicprocedure}{}
\algrenewcommand{\algorithmicthen}{}

% ``do { ... } while (cond)''
\algdef{SE}[DOWHILE]{Do}{doWhile}{\algorithmicdo}[1]{\algorithmicwhile\ #1}%

% ``for (x in y ... z)''
\newcommand{\ForRange}[3]{\For{#1 \textbf{in} #2 \ \ldots \ #3}}

% these are common math formatting commands that aren't defined by default
\newcommand{\union}{\cup}
\newcommand{\isect}{\cap}
\newcommand{\ceil}[1]{\ensuremath \left\lceil #1 \right\rceil}
\newcommand{\floor}[1]{\ensuremath \left\lfloor #1 \right\rfloor}

%%%%%%%%%%%%%%%%%%%%%%%%%%%%%%%%%%%%%%%%%%%%%%%%%%%%%%%%%%%%%%%%%%%%%%

\begin{document}

% STUDENT: Your text goes here!
\begin{enumerate}
\setcounter{enumi}{1}
\item (25\%)
Consider the general recurrence
\[
T(n) = \left\{
\begin{array}{cl}
c_0                     & \textrm{if\ }      n = 1 \\
a T(\frac{n}{b}) + f(n) & \textrm{otherwise}
\end{array}
\right.
\]

\noindent
In class, we showed that this recurrence has solution
\[
T(n) = c_0 n^{\log_b a} + \sum_{k = 0}^{\log_b n - 1} a^k f(n/b^k) .
\]

Suppose now that we alter the recurrence so that it terminates not for
$n = 1$ but for $n = n_0$, for some constant $n_0 > 1$.  In other
words, we replace the base case by ``$c_0 \textrm{\ if\ } n =
n_0$.'' You may assume for simplicity that $n$ and $n_0$ are both
powers of $b$.


\begin{enumerate}
\item
Thinking about the recursion tree for the revised recurrence, it isn't
surprising that its solution has the form
\[
T'(n) = c_0 a^{x} + \sum_{k = 0}^{x - 1} a^k f(n/b^k) .
\]

\noindent
What expression should replace the $x$ above?  Justify your answer.  You
might find it helpful to sketch the recursion tree for the revised
recurrence.
\\ \\
We will have that $x = \log(\frac{n}{n_0})$.
\begin{proof} A table for the recurrence $T(n)$ looks as follows: \\ \\
\begin{center}
\begin{tabular}{|c|c|c|c|}
\hline
 Level & Nodes & Work Per Node  \\
 \hline
 0& 1 &$f(n)$ \\
 1& $a$ &$f(\frac{n}{b})$ \\
 2& $a^2$ &$f(\frac{n}{b^2})$ \\
 k& $a^k$ &$f(\frac{n}{b^k})$\\
 $\log_b(\frac{n}{n_0})$ &$   a^{\log_b(\frac{n}{n_0})}   $& $c_0$ (new base case)  \\
 $\log_b(n)$ &$   a^{\log_b(n)}   $& $-$  \\
 \hline
\end{tabular}
\end{center}
Because our new base case is $n=n_0$, we only care about the levels above $\log_b(\frac{n}{n_0})$.  Therefore, we will sum from $ 0 \to \log_b(\frac{n}{n_0})$, so the recurrence $T'(n)$ becomes:
\begin{align*}
T'(n) = c_0a^{\log_b(\frac{n}{n_0})} + \sum_{k=0}^{\log_b(\frac{n}{n_0})-1} a^k f\bigl( \frac{n}{b^k} \bigr)
\end{align*}
\end{proof}
\item Show that $c_0 a^x$ above is still a constant times $n^{\log_b a}$.
\begin{proof}
We have $n_0$ fixed, so $a^{\log_b(n_0)}$ is fixed and not dependent on $n$.  Therefore, using log rules, we have:
\begin{align*}
c_0a^{\log_b(\frac{n}{n_0})} &= c_0 a^{\log_b(n)- \log_b(n_0)} \\
                             &= c_0 \bigl( a^{\log_b(n)} a^{-\log_b(n_0)} \bigr) \\
                             &= c'  \bigl(a^{\log_b(n)} \bigr) \\
                             &= c' \bigl(n^{\log_b(a)} \bigr)
\end{align*}
where $c'= c_0a^{-\log_b(n_0)}$, which is  a constant.
\end{proof}
\item 
The summation in (a) above can be rewritten as
\[
\sum_{k = 0}^{\log_b n - 1} a^k f(n/b^k) - 
  \sum_{k = \log_b n - y}^{\log_b n - 1} a^k f(n/b^k) 
\]

\noindent
for some constant $y$ independent of $n$.  What is this $y$?\\ \\
$y=\log_b(n_0)$.
\begin{proof}
This is evident from our use of logarithm rules.  $\log_b(\frac{n}{n_0}) = \log_b(n) - \log_b(n_0)$, therefore we can rewrite the summation in \textbf{(a)} to be:
\begin{align*}
\sum_{k=0}^{\log_b(\frac{n}{n_0})-1} a^k f\bigl( \frac{n}{b^k} \bigr) = \sum_{k = 0}^{\log_b n - 1} a^k f(n/b^k) - 
  \sum_{k = \log_b n - \log_b n_0}^{\log_b n - 1} a^k f(n/b^k) 
\end{align*}
This is also evident, because we wish to remove all the terms between levels $\log_b(\frac{n}{n_0})$ and $\log_b(n)$.  That is, for all levels smaller than the base case $n_0$.
\end{proof}
\item Show that the right-hand summation in (c) can be simplified to $c'
n^{\log_b a}$ where $c'$ is an expression independent of $n$.  
\begin{proof} We wish to show that:
\begin{align*}
\sum_{k = \log_b n - \log_b n_0 }^{\log_b n - 1} a^k f(\frac{n}{b^k}) = c'n^{\log_b a}
\end{align*}
for some $c'$ not dependent on $n$.  First, we need to adjust the bounds of our sum.  Notice that each $k$ will be of the form $k = \log_b n - \log_b n_0 +i$ where $i$ is an integer that ranges from $0 \to \log_b n_0-1$.  Therefore, we can rewrite this sum as: 
\begin{align*}
\sum_{k = \log_b n - \log_b n_0 }^{\log_b n - 1} a^k f(\frac{n}{b^k}) &= 
\sum_{i=0}^{\log_b n_0 -1} a^{\log_b n - \log_b n_0 +i }f\bigl(\frac{n}{b^{\log_b n -\log_b n_0 +i}}\bigr)\\
&= \sum_{i=0}^{\log_b n_0 -1}a^{\log_b(n)}a^{-\log_b(n_0)}a^i f\bigl(\frac{n}{b^{\log_b n}b^{-\log_b n_0} b^i}\bigr) \\
&= a^{\log_b(n)}a^{-\log_b(n_0)} \sum_{i=0}^{-\log_b n_0 -1}a^i f \bigl(\frac{n}{n n_0^{-1} b^i}\bigr) \\
&= a^{\log_b(n)}a^{-\log_b(n_0)} \sum_{i=0}^{-\log_b n_0 -1}a^i f \bigl(\frac{n_0}{b^i}\bigr) \\
&= c'a^{\log_b(n)}
\end{align*}
where
\begin{align*}
c'= a^{-\log_b(n_0)} \sum_{i=0}^{\log_b n_0 -1}a^i f \bigl(\frac{n_0}{b^i}\bigr)
\end{align*}
which is independent of $n$.
\end{proof}
\end{enumerate}
\end{enumerate}
\end{document}
