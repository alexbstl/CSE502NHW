%
% CSE Electronic Homework Template
% Last modified 4/20/2015 by Jeremy Buhler

\documentclass[11pt]{article}
\usepackage[left=0.7in,right=0.7in,top=0.7in,bottom=0.7in]{geometry}
\usepackage{fancyhdr} % for header
\usepackage{graphicx} % for figures
\usepackage{amsmath}  % for extended math markup
\usepackage{amssymb}
\usepackage[bookmarks=false]{hyperref} % for URL embedding
\usepackage[noend]{algpseudocode} % for pseudocode

%%%%%%%%%%%%%%%%%%%%%%%%%%%%%%%%%%%%%%%%%%%%%%%%%%%%%%%%%%%%%%%%%%%%%%
% STUDENT: modify the following fields to reflect your
% name/WUSTL Key, the current homework, and the current problem number

% Example: 
%\newcommand{\StudentName}{Jeremy Buhler}
%\newcommand{\WUSTLKey}{jbuhler}

\newcommand{\StudentName}{Alex Bernstein}
\newcommand{\WUSTLKey}{a.bernstein}
\newcommand{\HomeworkNumber}{2}
\newcommand{\ProblemNumber}{1}

%%%%%%%%%%%%%%%%%%%%%%%%%%%%%%%%%%%%%%%%%%%%%%%%%%%%%%%%%%%%%%%%%%%%%%%%
% You can pretty much leave the stuff up to the next line of %%'s alone.

% create header and footer for every page
\pagestyle{fancy}
\fancyhf{}
\lhead{\textbf{\StudentName{} (\WUSTLKey)}}
\chead{\textbf{Homework \HomeworkNumber}}
\rhead{\textbf{Problem \ProblemNumber}}
\cfoot{\thepage}

% preferred pseudocode style
\algrenewcommand{\algorithmicprocedure}{}
\algrenewcommand{\algorithmicthen}{}

% ``do { ... } while (cond)''
\algdef{SE}[DOWHILE]{Do}{doWhile}{\algorithmicdo}[1]{\algorithmicwhile\ #1}%

% ``for (x in y ... z)''
\newcommand{\ForRange}[3]{\For{#1 \textbf{in} #2 \ \ldots \ #3}}

% these are common math formatting commands that aren't defined by default
\newcommand{\union}{\cup}
\newcommand{\isect}{\cap}
\newcommand{\ceil}[1]{\ensuremath \left\lceil #1 \right\rceil}
\newcommand{\floor}[1]{\ensuremath \left\lfloor #1 \right\rfloor}

%%%%%%%%%%%%%%%%%%%%%%%%%%%%%%%%%%%%%%%%%%%%%%%%%%%%%%%%%%%%%%%%%%%%%%

\begin{document}

% STUDENT: Your text goes here!
\begin{enumerate}
\setcounter{enumi}{3}
\item (25\%)
Suppose you have an empty hash table $T$ with $m = 8$ slots that stores
integer records (i.e.\ the record and its key are one and the same).
The table resolves collisions by open addressing with double hashing,
and its hash functions are $h_1(k) = (5 \cdot k) \bmod 8$ and
$h_2(k) = (3 \cdot k) \bmod 7 + 1$.  (\textit{Disclaimer}: these are
\emph{not} good hash functions, just examples to help you reinforce
your understanding of open-addressed hashing.)

\begin{enumerate}
\item

Sketch diagrams showing the state of the hash table and give its load
factor after \emph{each} of the following series of calls:
\begin{enumerate}
\item 
\begin{verbatim}
T.insert(293)
\end{verbatim}
\begin{center}
\begin{tabular}{|c | c|}
\hline
Index & Slot Value\\
\hline
0 & \\
\hline
1 & * \\
\hline
2 & \\
\hline
3 & \\
\hline
4 & \\
\hline
5 & \\
\hline
6 & \\
\hline
7 &  \\
\hline
\end{tabular}
\end{center}
Load Factor = $\frac{1}{8}$
\item
\begin{verbatim}
T.insert(30181)
\end{verbatim}
\begin{center}
\begin{tabular}{|c | c|}
\hline
Index & Slot Value\\
\hline
0 & \\
\hline
1 & * \\
\hline
2 & \\
\hline
3 & \\
\hline
4 & \\
\hline
5 & \\
\hline
6 & \\
\hline
7 & * \\
\hline
\end{tabular}
\end{center}
Load Factor = $\frac{2}{8}$
\item
\begin{verbatim}
T.insert(38388)
\end{verbatim}
\begin{center}
\begin{tabular}{|c | c|}
\hline
Index & Slot Value\\
\hline
0 & \\
\hline
1 & *\\
\hline
2 & \\
\hline
3 & \\
\hline
4 & *\\
\hline
5 & \\
\hline
6 & \\
\hline
7 & *\\
\hline
\end{tabular}
\end{center}
Load Factor = $\frac{3}{8}$
\item
\begin{verbatim}
T.insert(62)
\end{verbatim}
\begin{center}
\begin{tabular}{|c | c|}
\hline
Index & Slot Value\\
\hline
0 & \\
\hline
1 & *\\
\hline
2 & \\
\hline
3 & \\
\hline
4 & *\\
\hline
5 & \\
\hline
6 & *\\
\hline
7 & *\\
\hline
\end{tabular}
\end{center}
Load Factor = $\frac{4}{8}$
\item
\begin{verbatim}
T.insert(3421)
\end{verbatim}
\begin{center}
\begin{tabular}{|c | c|}
\hline
Index & Slot Value\\
\hline
0 & \\
\hline
1 & *\\
\hline
2 & \\
\hline
3 & *\\
\hline
4 & *\\
\hline
5 & \\
\hline
6 & *\\
\hline
7 & *\\
\hline
\end{tabular}
\end{center}
Load Factor = $\frac{5}{8}$
\\
\item
\begin{verbatim}
T.delete(293)
\end{verbatim}
\begin{center}
\begin{tabular}{|c | c|}
\hline
Index & Slot Value\\
\hline
0 & \\
\hline
1 & \texttt{deleted}\\
\hline
2 & \\
\hline
3 & *\\
\hline
4 & *\\
\hline
5 & \\
\hline
6 & *\\
\hline
7 & *\\
\hline
\end{tabular}
\end{center}
Load Factor = $\frac{4}{8}$
\\
\end{enumerate}

\item 
After the hash table $T$ has performed the above operations, what is the
sequence of slots in the table inspected by each of the following
\texttt{find} operations before it returns?

\begin{enumerate}


\item
\begin{verbatim}
T.find(293)
\end{verbatim}
$h_1(293) = 1$, and $h_2(k) = 5$, so this operation will check, in order, Slots: $1 \to 6 \to 3 \to 0 \to 5 \to 2 \to 7 \to 4$.  Because slot 1 is \texttt{deleted}, it will look at the next slot, and therefore return whatever is in slot $6$.
\item
\begin{verbatim}
T.find(38924)
\end{verbatim}
$h_1(38924) = 4$, and $h_2(38924) = 6$.  Because Slot $4$ is full from above, it will return the item stored in that slot  The order of slots is: $4 \to 2 \to 0 \to 6 \to 4...$.
\item
\begin{verbatim}
T.find(62)
\end{verbatim}
$h_1(62) = 6$ and $h_2(62) = 5$.  The order of slots is therefore: $6 \to 3 \to 0 \to 5 \to 2 \to 7 \to 4 \to 1$.  Because slot $6$ has an item, it will return whatever is in that slot.
\end{enumerate}
\item
Just to show that the above table and hash function parameters aren't
great, give an example of a key for which the \texttt{insert} operation
could fail even if the table is not full.  Explain why insertion could
fail for this key.

We can use the example above, where will \texttt{T.insert(38924)} fail even if slots 1, 3, 5, and 7 are open if the remaining slots are filled.  In this case $h_1(38924) = 4$, and $h_2(38924) = 6$.  Insert will fail if slots 0, 2, 4, and 6 are filled because $h_2(38924)$ shares a common divisor with $8$, $h_1(38924) + i*h_2(38924)$ will form a cyclic subgroup of $\mathbb{Z}\bmod8$, and thus will not reach every slot.
\end{enumerate}
\end{enumerate}
\end{document}
